%%%%%%%%%%%%%%%%%%%%%%%%%%%%%%%%%%%%%%%%%%%%%%%%%%%%%%%%%%%%%%%%%%%%%%%%%%%%%%
%% Upravená původní dokumentace od Davida Martinka.
%%%%%%%%%%%%%%%%%%%%%%%%%%%%%%%%%%%%%%%%%%%%%%%%%%%%%%%%%%%%%%%%%%%%%%%%%%%%%%
\documentclass[12pt,a4paper,titlepage,final]{article}

% cestina a fonty
\usepackage[utf8]{inputenc}
% balicky pro odkazy
\usepackage[bookmarksopen,colorlinks,plainpages=false,urlcolor=blue,unicode]{hyperref}
\usepackage{url}
\usepackage{amsmath}
\usepackage{capt-of}
\usepackage[Q=yes]{examplep}
\usepackage{enumitem}
\usepackage{varwidth}

% obrazky
\usepackage{graphicx}
% velikost stranky
\usepackage[text={15.2cm, 25cm}, ignorefoot]{geometry}

\begin{document}
%%%%%%%%%%%%%%%%%%%%%%%%%%%%%%%%%%%%%%%%%%%%%%%%%%%%%%%%%%%%%%%%%%%%%%%%%%%%%%
% titulní strana
\begin{titlepage}

\vspace*{2cm}
\begin{figure}[!h]
  \centering
  \includegraphics[width=10cm]{img/FIT_color_CMYK_EN.eps}
\end{figure}

\vfill

\begin{center}
\begin{Huge}
Simple DNS
\end{Huge}
\end{center}

\vfill

\begin{center}
\begin{Large}
November 20, 2016
\end{Large}
\end{center}

\vfill

\begin{flushleft}
\begin{normalsize}
\begin{tabular}{ll}
\bf Author:\hspace{3mm} & Marek Bielik (\url{mail@marek.onl}) \\[3mm]
& Faculty of Information Technology \\
& Brno University of Technology
\end{tabular}
\end{normalsize}
\end{flushleft}
\end{titlepage}


%%%%%%%%%%%%%%%%%%%%%%%%%%%%%%%%%%%%%%%%%%%%%%%%%%%%%%%%%%%%%%%%%%%%%%%%%%%%%%
% obsah
\pagestyle{plain}
\pagenumbering{roman}
\setcounter{page}{1}
\tableofcontents

%%%%%%%%%%%%%%%%%%%%%%%%%%%%%%%%%%%%%%%%%%%%%%%%%%%%%%%%%%%%%%%%%%%%%%%%%%%%%%
% textova zprava
\newpage
\pagestyle{plain}
\pagenumbering{arabic}
\setcounter{page}{1}

%%%%%%%%%%%%%%%%%%%%%%%%%%%%%%%%%%%%%%%%%%%%%%%%%%%%%%%%%%%%%%%%%%%%%%%%%%%%%%
\section{Overview} 
%%%%%%%%%%%%%%%%%%%%%%%%%%%%%%%%%%%%%%%%%%%%%%%%%%%%%%%%%%%%%%%%%%%%%%%%%%%%%%
RoughDNS is a recursive, concurrent DNS server that supports various resource
record types and runs over UDP on Unix-like systems. It is written in C++
according to RFC 1035~\cite{rfc1035} and it also supports loading of multiple
zone files by utilizing~\cite{ldns}).

%%%%%%%%%%%%%%%%%%%%%%%%%%%%%%%%%%%%%%%%%%%%%%%%%%%%%%%%%%%%%%%%%%%%%%%%%%%%%%
\section{Features} \label{sec:features}
%%%%%%%%%%%%%%%%%%%%%%%%%%%%%%%%%%%%%%%%%%%%%%%%%%%%%%%%%%%%%%%%%%%%%%%%%%%%%%
\subsection{Usage}
\begin{verbatim}
	./roughDNS [-m <ip_addr>] [-h] [-p <port#>] [<zonefile1>] [<zonefileX>]
\end{verbatim}
\begin{itemize}
	\item
		\begin{verbatim}
		-m [--mitm] <ip_address>
		\end{verbatim}
		Responds to every A and MX query with the given IP address. This parameter
    simulates the man in the middle attack. Other records are served in the
    standard way.
	\item 
		\begin{verbatim}
		-h [--help]
		\end{verbatim}
		Prints out a simple user manual.
	\item
		\begin{verbatim}
		-p [--port] <port number>
		\end{verbatim}
		The server will bind its socket with the given port number. If not
    specified, the standard port number will be used - 53.
\end{itemize}
The last optional parameters are the paths to zone files. The resource records
specified in the zone files must be unique.

\subsection{Supported resource record types}
\begin{center}
	\begin{varwidth}[t]{.5\textwidth}
		\begin{itemize}
			\item A
			\item MX
			\item SOA
			\item PTR
		\end{itemize}
	\end{varwidth}
	\hspace{4em}
	\begin{varwidth}[t]{.5\textwidth}
		\begin{itemize}
			\item AAAA
			\item CNAME
			\item NS
			\item TXT
		\end{itemize}
	\end{varwidth}
\end{center}

\subsection{Description of the supported resource records}

The resource records which the server is authoritative for can be loaded from a
zone file. The server also supports loading of multiple zone files. In this
case, the records in every zone must be unique. The user can specify arbitrary
number of the zone files to be loaded by typing the paths as the last parameters
on the command line while launching the server.

When the server doesn't find any record for the requested domain name, it
resolves the client's query recursively and sends back a non-authoritative
answer. The server also supports the Authority and Additional sections of a DNS
message.

When the server receives a query or sends a response, it also prints the
information to the standard output. Examples are shown in the
section~\ref{sec:examples}. If the answer is empty, the server won't print
anything but the question.

If the client is looking for a record which happens to be 'behind' a CNAME
record in the zone file, the server will look for every occurence of the actual
canonical name and send both the desired record as well as the CNAME record.
Recursive CNAME records are also supported. Example~\ref{subsec:ex2} shows this
case.

%%%%%%%%%%%%%%%%%%%%%%%%%%%%%%%%%%%%%%%%%%%%%%%%%%%%%%%%%%%%%%%%%%%%%%%%%%%%%%
\section{Implementation} \label{sec:implementation}
%%%%%%%%%%%%%%%%%%%%%%%%%%%%%%%%%%%%%%%%%%%%%%%%%%%%%%%%%%%%%%%%%%%%%%%%%%%%%%
The server is written in C++ and it makes use of an open source
library~\cite{ldns} in order to parse the zone files and to resolve the
non-authoritative requests. It is supposed to run on Unix-like systems providing
the library has been installed. The server also utilizes Berkeley sockets for
UDP communication with clients which is done manually without utilizing any
third-party libraries. It supports only the IN class of resource records and
only one question per message.

In order to keep the length of the message under the limited size
(512B)~\cite{rfc1035} section 2.3.3, the server supports the compression scheme
of domain names introduced in RFC 1035 \cite{rfc1035} section 4.1.4. The
compression eliminates the repetition of domain names in a message. In the case
that the length of a message exceeds the maximum length, the server will
truncate the message and set the truncation bit.

Since it is a concurrent server, it forks a new process for every new incoming
message and processes it separately.

The server can be terminated by receiving the SIGINT signal. It has been tested
in Valgrind for potential memory leaks and none were present (even in the forked
processes).

%%%%%%%%%%%%%%%%%%%%%%%%%%%%%%%%%%%%%%%%%%%%%%%%%%%%%%%%%%%%%%%%%%%%%%%%%%%%%%
\section{Examples} \label{sec:examples}
%%%%%%%%%%%%%%%%%%%%%%%%%%%%%%%%%%%%%%%%%%%%%%%%%%%%%%%%%%%%%%%%%%%%%%%%%%%%%%

\subsection{Example1:} \label{subsec:ex1}

\begin{verbatim}
	q: foo: type SOA, class IN
	r: foo: type SOA, class IN, mname a.root-servers.net.
	q: foo: type A, class IN
	r: foo: type A, class IN, addr 85.214.123.64
	r: foo: type A, class IN, addr 85.214.123.63
	q: foo: type MX, class IN
	r: foo: type MX, class IN, preference 50, mx mx1.foo.
	q: www.foo: type AAAA, class IN
	r: www.foo: type AAAA, class IN, addr 2001:67c:1220:809::93e5:917
	q: 85.214.123.64.in-addr.arpa: type PTR, class IN
	r: 85.214.123.64.in-addr.arpa: type PTR, class IN, www.foo.foo.
	q: foo: type NS, class IN
	r: foo: type NS, class IN, ns l.root-servers.net.
	q: txt.foo: type TXT, class IN
	r: txt.foo: type TXT, class IN, txt "foobar"
\end{verbatim}
In this example we can see the standard output for the supported resource
records. There is a SOA, A, MX, AAAA, PTR, NS and a TXT record listed
respectively. \texttt{q} stands for a question and \texttt{r} stands for a
response.

\subsection{Example2:} \label{subsec:ex2}

\begin{verbatim}
	q: www.foo.ex: type A, class IN
	r: www.foo.ex: type CNAME, class IN, cname bar.foo.ex
	r: bar.foo.ex: type CNAME, class IN, cname foo.ex
	r: foo.ex: type A, class IN, addr 85.214.123.64
	r: foo.ex: type A, class IN, addr 85.214.123.63
\end{verbatim}
The client is asking for an A record of the \texttt{www.foo.ex} domain name. We
can see that there is one CNAME record for this name with the canonical name
\texttt{bar.foo.ex} and that there's also another CNAME record for this name as
well. After these two CNAME records, there are two A records which satisfy the
original request.

\newpage

\bibliographystyle{czechiso}

\begin{flushleft}
	\bibliography{refs}
\end{flushleft}
\end{document}
